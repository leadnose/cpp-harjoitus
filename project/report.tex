% Created 2012-12-07 Fri 12:52
\documentclass[11pt]{article}
\usepackage[utf8]{inputenc}
\usepackage[T1]{fontenc}
\usepackage{graphicx}
\usepackage{longtable}
\usepackage{float}
\usepackage{wrapfig}
\usepackage{soul}
\usepackage{amssymb}
%% \usepackage{hyperref}


\title{C++ project report}
\author{Janne Ronkonen}
\date{07 December 2012}

\begin{document}

\maketitle

\setcounter{tocdepth}{3}
%% \tableofcontents
\vspace*{1cm}


\section{Design decisions}
\label{sec-2}

\subsection{The class itself}
\label{sec-2.1}

The class members consists of char pointer to the beginning of the
buffer (m\_buf) that holds the actual contents of the string, and two
size\_t:s that are used to keep track of the size of the buffer
(m\_bufsize), and the logical size of the string (m\_used).  The m\_used
may be null if the string has no characters. In hindsight it would've
probably been more consistent to never allow m\_buf to be null.

\subsection{The testdriver}
\label{sec-2.2}

The testdriver consists of a bunch of functions that take no arguments
and return no value. A test is succesful if the function returns
normally, and test fails if the function throws an exception. The
driver's main() -function calls each of the test-functions and records
how many of them failed and succeeded, and also records the names of
the functions that failed, and prints this info into std::cout.


\section{Challenges during the project}
\subsection{Getting it to compile}

One problem was of course getting the whole thing to compile in the
first place. Error messages from the g++ compiler aren't always very helpful, and
finding out the ``true'' cause of an  error that prevents compilation
isn't always as trivial as one might think it is.

\subsection{Keeping track of dynamically allocated memory correctly}

One of the more difficult things was keeping track of the dynamically
allocated buffers of memory and not leaking any memory. Thankfully C++
idioms such as RAII and using constructors, destructors and
assignment-operator make this task somewhat simpler. One particular
problem regarding memory was the resizing of the buffer found in
push\_back() and pop\_back() -methods.

\subsection{Implementation of the IO-operators}

Because of the statefulness of IO-operations, I found them to be the
most challenging to implement. 

\subsection{Maintaining separate header and implementation files by hand}

Because the project-specification required separate header and
implementation files, it became something of a problem to keep the
header and implementation in synch by hand. Since this project is of
trivial size, I figure there must be some better method of doing this
sort of mundane task in bigger projects that may have orders of
magnitude more files. 

\subsection{Testing}

Since I was both the tester and the implementor of the class, I found
it somewhat difficult to write good tests since I already was thinking
how I would implement the tested functionality, and this in turn
influenced how and what I would test. I feel that one cannot truly
write comprehensive and exacting tests for his/her own code because of
this phenomenom, and it would be better to separate these
responsibilities in bigger and/or more important projects.

\end{document}
